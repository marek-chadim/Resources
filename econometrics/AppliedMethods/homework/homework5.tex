\documentclass[11pt, a4paper]{article}
%\usepackage{geometry}
\usepackage[inner=1.5cm,outer=1.5cm,top=2.5cm,bottom=2.5cm]{geometry}
\pagestyle{empty}
\usepackage{graphicx}
\usepackage{amsmath, amssymb}
\usepackage[usenames,dvipsnames]{color}
\definecolor{darkblue}{rgb}{0,0,.6}
\definecolor{darkred}{rgb}{.7,0,0}
\definecolor{darkgreen}{rgb}{0,.6,0}
\definecolor{red}{rgb}{.98,0,0}
\usepackage[colorlinks,pagebackref,pdfusetitle,urlcolor=darkblue,citecolor=darkblue,linkcolor=darkred,bookmarksnumbered,plainpages=false]{hyperref}
\renewcommand{\thefootnote}{\fnsymbol{footnote}}



\thispagestyle{plain}

%%%%%%%%%%%% LISTING %%%
\usepackage{listings}
\usepackage{caption}
\DeclareCaptionFont{white}{\color{white}}
\DeclareCaptionFormat{listing}{\colorbox{gray}{\parbox{\textwidth}{#1#2#3}}}
\captionsetup[lstlisting]{format=listing,labelfont=white,textfont=white}
\usepackage{verbatim} % used to display code
\usepackage{fancyvrb}
\usepackage{acronym}
\usepackage{amsthm}
\VerbatimFootnotes % Required, otherwise verbatim does not work in footnotes!



\definecolor{OliveGreen}{cmyk}{0.64,0,0.95,0.40}
\definecolor{CadetBlue}{cmyk}{0.62,0.57,0.23,0}
\definecolor{lightlightgray}{gray}{0.93}



\lstset{
%language=bash,                          % Code langugage
basicstyle=\ttfamily,                   % Code font, Examples: \footnotesize, \ttfamily
keywordstyle=\color{OliveGreen},        % Keywords font ('*' = uppercase)
commentstyle=\color{gray},              % Comments font
numbers=left,                           % Line nums position
numberstyle=\tiny,                      % Line-numbers fonts
stepnumber=1,                           % Step between two line-numbers
numbersep=5pt,                          % How far are line-numbers from code
backgroundcolor=\color{lightlightgray}, % Choose background color
frame=none,                             % A frame around the code
tabsize=2,                              % Default tab size
captionpos=t,                           % Caption-position = bottom
breaklines=true,                        % Automatic line breaking?
breakatwhitespace=false,                % Automatic breaks only at whitespace?
showspaces=false,                       % Dont make spaces visible
showtabs=false,                         % Dont make tabls visible
columns=flexible,                       % Column format
morekeywords={__global__, __device__},  % CUDA specific keywords
}

%%%%%%%%%%%%%%%%%%%%%%%%%%%%%%%%%%%%
\begin{document}
\begin{center}
  {\Large \textsc{Problem Set 5}}

  MGMT 737
\end{center}

\begin{enumerate}
  %%% TO DO -- WRITE A DEMAND FUNCTION QUESTION
\item \textbf{Poisson Regression}. This analysis will use the dataset
  \texttt{Detroit.csv}. You should use built-in pacakges for this --
  in \texttt{R}, use the \texttt{fixest} library. In \texttt{Stata},
  use the \texttt{reghdfe} and \texttt{ppmlhdfe} packages.
  \begin{enumerate}
  \item Run an OLS regression of \texttt{flows} (the number of workers
    who work in \texttt{home\_ID} and work in \texttt{work\_ID}) on
    \texttt{distance\_Google\_miles}, and include \texttt{home\_ID} and
    \texttt{work\_ID} fixed effects (absorb them), and cluster on
    \texttt{home\_ID}. Report the coefficient and standard error on
    \texttt{distance\_Google\_miles}.
  \item Run an OLS regression of \texttt{log(flows)} on
    \texttt{log(distance\_Google\_miles)} and include \texttt{home\_ID}
    and \texttt{work\_ID} fixed effects , omitting the cells with zero
    flows. Cluster on \texttt{home\_ID}. Report the coefficient and
    standard error on \texttt{log(distance\_Google\_miles)}.
  \item Repeat part 1b, but instead of omitting the zero cells, run
    the OLS regression of \texttt{log(c + flows)} for c = 0.1, 1 and
    10. Compare how your coefficients change. 
  \item Finally, repeat part 1a using Poisson regression, and contrast
    the estimates to Part b and c.
  \end{enumerate}
\item \textbf{Duration Modeling}. This analysis will use the dataset
  \texttt{acs\_duration.csv}. The
  \texttt{acs\_duration.csv} dataset is from the American Community
  Survey in 2019, and has heads of households' responses to the
  question ``How long have you lived in this home?''
  (\texttt{moving\_approx}  -- in reality, this value is given as
    a range in the public data -- I have imputed using the midpoint. A
    fun exercise left to the reader is to think about how to
    generalize this problem using ranges.) and homeownership
  (\texttt{homeowner} vs. \texttt{renter}).
    \begin{enumerate}
    \item Using the ACS data,write down how to estimate the unconditional probability that a household stays in a home for T or more years, using the available data. Estimate this for $T=7$ and report the value.
    \item Calculate the hazard rate for each observed value of
      \texttt{moving\_approx}. Report this value for $T=7$.
    \item Recalculate these hazard values for $T=7$ for homeowners and
      renters. Contrast the difference in hazard rates over time.
    \end{enumerate}
  \end{enumerate}


\end{document}