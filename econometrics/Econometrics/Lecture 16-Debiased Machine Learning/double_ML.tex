%% See: https://bookdown.org/yihui/rmarkdown-cookbook/multi-column.html
%% I've made some additional adjustments based on my own preferences (e.g. cols
%% should be top-aligned in case of uneven vertical length)
\newenvironment{columns}[1][]{}{}

\newenvironment{column}[1]{\begin{minipage}[t]{#1}\ignorespaces}{%
\end{minipage}
\ifhmode\unskip\fi
\aftergroup\useignorespacesandallpars
}

\def\useignorespacesandallpars#1\ignorespaces\fi{%
#1\fi\ignorespacesandallpars}

\makeatletter
\def\ignorespacesandallpars{%
  \@ifnextchar\par
    {\expandafter\ignorespacesandallpars\@gobble}%
    {}%
}
\makeatother



% \usepackage{slashbox}
\title{ Double/Debiased Machine Learning}
\author{Chris Conlon }
\institute{NYU Stern }



\date{\today}

\begin{document}
\maketitle

\section*{Setup}

\begin{frame}{The core problem}
\begin{align*}
Y_i &= \alert{\tau_i} \cdot D_i + g_0(X_i) + u_i &\text{ with }\E[u_i \mid x_i, D_i]=0\\
D_i &= m_0(X_i) + v_i &\text{ with } \E[v_i \mid x_i ]=0
\end{align*}
\begin{itemize}
    \item $D_i$ is \alert{treatment indicator} and $\tau_i$  is \alert{treatment effect}
    \item $X_i$ are covariates (``controls'' or ``confounders'') $\rightarrow$
    \item We call $m_0(\cdot)$ and $g_0(\cdot)$ \alert{nuisance parameters} or \alert{nuisance functions}.
\end{itemize}
\end{frame}

\begin{frame}{What if we try ``machine learning?`'}
\begin{align*}
Y_i &= \alert{\tau_i} \cdot D_i + g_0(X_i) + u_i &\text{ with }\E[u_i \mid x_i, D_i]=0\\
D_i &= m_0(X_i) + v_i &\text{ with } \E[v_i \mid x_i ]=0
\end{align*}
Could alternate steps:
\begin{enumerate}
    \item Fit random forest of $Y_i - \widehat{\tau_i} \cdot D_i$ on $Z_i$ to get $\widehat{g_0}(Z_i)$.
    \item Run OLS of $Y_i - \widehat{g_0}(Z_i)$ on $D_i$ to get $\widehat{\tau}$ or $\widehat{\tau_i}$
\end{enumerate}
This fits the data great but gives terrible estimates of $\widehat{\tau}$!\\
Why? Frisch-Lovell-Waugh really needs things to be \alert{linear}!
\end{frame}

\begin{frame}{What goes wrong}
\begin{itemize}
\item We are trading off \alert{bias} for \alert{variance reduction}
\item But, we can't trust \alert{plug-in} estimates when $m_0(\cdot),g_0(\cdot)$ are not linear.
\item So bias can be very dangerous now...
\end{itemize}
What we need to do is \alert{orthogonalize} things properly (like a nonlinear Frisch-Lovell-Waugh)
\begin{itemize}
\item Bias from \alert{regularization} \textrightarrow Orthogonalization
\item Bias from \alert{overfitting} \textrightarrow Sample Splitting
\end{itemize}
\end{frame}



\begin{frame}{Sample Splitting}

\begin{itemize}
\item Split the sample into two parts \alert{main sample} and \alert{auxiliary sample}.
\item On the auxiliary Sample:
\begin{itemize}
\item Estimate $\widehat{g}(X_i)$ from $Y_i = \tau \cdot D_i + g(X_i) + u_i $
\item Estimate $\widehat{m}(X_i)$ from $D_i =  m(X_i) + v_i $
\end{itemize}
\item Now on the main sample:
\begin{itemize}
\item Compute the residual:  $\widehat{v_i} = D_i - \widehat{m}(X_i)$.
\item Estimate $\hat{\tau}=\left(\hat{v}^{\prime} D\right)^{-1} \hat{v}^{\prime}(Y-\hat{g}(X))$
\end{itemize}
\end{itemize}

\end{frame}

\begin{frame}{More Info}
To learn more watch Chernozhukov lecture here:
\url{https://www.youtube.com/watch?v=eHOjmyoPCFU&t=37s}\\

The \texttt{R} package
\url{https://docs.doubleml.org/stable/index.html}
\end{frame}



\section*{Thanks!}




\end{document}